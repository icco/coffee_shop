\documentclass[11pt]{article}
\usepackage{setspace}
\usepackage{url}
\usepackage{geometry}
\geometry{letterpaper}
\begin{document}
\title{\vfill coffee\_shop: Investigations into modern word processors.} %\vfill gives us the black space at the top of the page
\author{
Nathaniel Welch, Dr. Clark Turner\vspace{10pt} \\
California State Polytechnic University\vspace{10pt} \\
}
\date{\today}
\maketitle

\doublespacing

\vfill
\begin{abstract}
My senior project was spent building a desktop application similar to WriteRoom and OmmWriter. These two applications are both word processors for the Macintosh OS X operating system. They take what is commonly described as a reset on word processing software, bringing the software back to the days of Microsoft Word 3.0 and Word Perfect. They do this by spending more time on focusing on the design of the software interface, and focusing on keeping it minimalistic, instead of filling the product with new obscure features, which is a common complaint against the current iterations of Microsoft Word.

The final application, named coffee\_shop, ended up not meeting my expectations. Having spent most of my education on developing applications for the internet, instead of the desktop, I ran into pitfalls which, if this had been an internet application wouldn't have been an issue.
\end{abstract}

\thispagestyle{empty}
% \newpage
% 
% \thispagestyle{empty} % Remove page number from TOC
% \tableofcontents

\newpage

\section{Introduction}

Depending on the job, people use different tools. Some tools are incredibly specialised, such as post hole diggers and PVC pipe warmers. Others are much more generic, such as hammers and cameras. Notice though, that in the hardware world, the majority of tools serve one job. That job, such as hammering, can be applied to a wide variety of ways (hammering in a nail, breaking apart structures, putting stakes in the ground). In the software world, historically tools have been built to be much more generic. Microsoft Word for example not only let you write documents, but also create spreadsheets, resumes and many other things. While this has let Microsoft make more money by selling their software to a wider variety of needs, it has created software that is hard to use and hard to maintain. To fight this, companies and individuals are starting to create software that have less features and are easier to use.

\section{Problem description}

The problem coffee\_shop tries to tackle is whether or not developing a desktop application focused on writing, instead of formating, is a good idea.

\section{Survey of relevant work}
\subsection{Professional Offerings}

\subsubsection{Microsoft Word}
\subsubsection{Pages}
\subsubsection{Open Office}

\subsection{Similar implementations}

\subsubsection{WriteRoom}

\subsubsection{OmmWriter}

\section{coffee\_shop}

\section{Evaluation of coffee\_shop}
\subsection{Development decisions}

I made some decisions while developing and designing coffee\_shop, which may be considered controversial or unintelligent. The two big decisions I made were to use Ruby as my programming language and Qt as my graphics framework.

\subsubsection{Ruby}

Ruby is an interesting beast. I selected it because I had little experience with the language and it had a very large community online.

\subsubsection{Qt}

\subsection{Features and their implementations}

\subsubsection{File writing}

\subsubsection{Pagination}

\subsubsection{Printing}

\subsubsection{Customization and preference storing}

Talk about how I did it (yaml) and how other programs do it.

\section{Conclusion}

\newpage
\nocite{*}
\bibliographystyle{IEEEannot}
\bibliography{texreport}
\end{document}
